\documentclass[UTF9]{ctexart}
\title{一种泛商业形态的商业创新模式及其在各商业领域的应用}
\date{2019年1月2日}
\author{未来工程科技\\邹远春}
\usepackage{xeCJK}
\usepackage{graphicx}
\usepackage{float}
\usepackage{listings}
\usepackage{verbatim}
\usepackage{mathtools}
\bibliographystyle{alpha}

\begin{document}

\maketitle

\newcommand\Emph{\textbf}

\tableofcontents

\section{新商业模式}

\subsection{名词解释}

\Emph{商业模式}
企业与企业之间、企业的部门之间、乃至与顾客之间、与渠道之间都存在各种各样的交易关系和连结方式称之为商业模式。它是一个企业满足消费者需求的系统,这个系统组织管理企业的各种资源(资金、原材料、人力资源、作业方式、销售方式、信息、品牌和知识产权、企业所处的环境、创新力,又称输入变量),形成能够提供消费者无法自力而必须购买的产品和服务(输出变量),因而具有自己能复制且别人不能复制,或者自己在复制中占据市场优势地位的特性。

\Emph{系统与体系结构}
系统是由相互作用相互依赖的若干组成部分结合而成的,具有特定功能的有机整体,而且这个有机整体又是它从属的更大系统的组成部分。运动着的若干部分,在相互联系、相互作用之中形成的具有某种确定功能的整体,谓之系统。体系结构提供了对组成系统的组件或构造块的描述以及这些组件间复杂的内部关系,是最重要、普遍性、高等级的战略性发明及决策,以及它们与整体结构(如:必备元素和关系)的相关原理、特性和行为。

\Emph{企业}
企业一般是指以盈利为目的,运用各种生产要素(土地、劳动力、资本、技术和企业家才能等),向市场提供商品或服务,实行自主经营、自负盈亏、独立核算的法人或其他社会经济组织。现代经济学理论认为,企业本质上是“一种资源配置的机制”,其能够实现整个社会经济资源的优化配置,降低整个社会的“交易成本”。

\Emph{平台经济}
所谓平台经济就是经营者建立一个平台,把有需要的利益相关方吸引到平台上合作,参与者获益的同时平台建立者也可以获得巨大的收益。

\Emph{Costco模式}
规模经济与分享经济的结合体,但同时将商品做到极好,将价格降到了极低,同时又将服务做到了超人预期。
\begin{itemize}
\item 精选商品极致低价、高性价比
\item 会员费成为主要收入来源
\item 消费简单、产品质量靠谱
\item 优质服务
\end{itemize}

\subsection{新商业模式核心}

\Emph{核心理念}
\begin{itemize}
\item 企业转换成平台,它是价值放大者,平台成就价值
\item 企业家、股东或各组织领导者转换成价值整合者,依靠配置社会资源,间接创造财富
\item 所有权:归股东(生产资料提供者)所有转换成归所有生态协作者所有
\item 组织结构从树状结构转化为扁平的或网状结构
\item 收益透明,生态参与者共同享有分配权
\item 生产关系的改变,你的价值在于你的劳动价值,而不仅仅是工资。价值回归生态价值创造者本身,新的激励与治理模式,雇佣关系转变,以前是按照员工劳动力贡献分配,现在加入员工创造价值分配,员工转换成服务提供者,依靠个体劳动创造直接财富,消费者转换成直接价值提供者,消费带动行业发展,同时也可以作为营销推广者,比如口碑宣传、粉丝社群等模式,都是生态的参与者与贡献者,享有共同的分配权。
\item 深入践行Costco模式
\item 开放力,人在一种开放的环境之下,是有可能构建一种有生命力的、可扩展的秩序的;平台开放不仅仅是说开放资源,而是整个平台的生态参与者都可以参与平台的规则、决策、制度等建设,决定平台的发展方向。
\item 基于诚信构建商业生态,降低社会运行成本
\item 未来个人就是独立的经济体
\end{itemize}

\subsection{新商业模式目标}
\begin{itemize}
\item 生产关系变革,带来更公平的分配,良性的生态循环(钱花出去但又能赚回来,减少某些人或团体财富剧增的貔貅效应,减少贫富差距,让大多数消费者敢于消费,勇于消费,从而保障消费,带动经济发展)
\item 减少消费者的单一消费支出,但能带来更多的消费,提升生活水平
\item 打造透明、开放、良好的社会诚信体系,降低社会运行成本
\end{itemize}

%公开 不仅更开放透明的投资,也让大家作为统一体来共同协商发展 开放力:传统平台归公司所有,平台的服务者与消费者等利益相关方都没有权利、冲突、卸磨杀驴等等。
%平台经济 线上线下结合 快递物流 移动快捷支付 直播短视频等等让销售产生巨大的变化
%营销模式创新:平台、广告、促销、直播、短视频/品牌植入、互联网广告、平面广告、广播电视媒体
\section{新商业模式基于区块链技术的实践}

区块链是一个信任的机器,是在完全不信任的节点之间建立信任机制的技术,是利用互联网传递价值的一种价值网络。

商业三流(信息流、资金流、物流),特别是信息流和资金流,基于区块链去信任化和价值传输的特性,可以在区块链上高效的流转。

\subsection{以太坊POA}

系统DAPP建立在类似POA Network的以太坊POA网络上,以太坊满足一下几点:
\begin{itemize}
\item 图灵完备的智能合约,设计了编程语言 Solidity 和虚拟机 EVM
\item 采用账户系统和世界状态,而不是 UTXO,容易支持更复杂的业务逻辑
\item 基于POA的共识算法满足一定的延展性和性能
\item 生态开放,社区活跃
\item 技术成熟度高,下一步的POS共识、分片、零知识证明、Plasma、VDF等技术的融入会使得其性能和延展性更加强大
\end{itemize}

系统可以免费分发Gas给系统内的用户来执行智能合约。

\subsection{Token}

\begin{itemize}

\item 基于以太坊ERC20标准,发行平台激励Token。

\item 为了规避法律风险,Token只是代表分配权,不能直接参与法币交易。

\item 我们通过构建基于监管的可升级的智能合约来发行对法币1:1的数字资产,方便法币在系统内的流通(为了规避法律风险,该数字资产不能随意转账,只有符合平台监管的智能合约才能交易与转账)。当用户通过移动支付等方式支付一笔时,相应的发行同等额度的数字资产到支付用户的区块链账户里;当用户数字资产退出系统时,系统通过银行或第三方支付转账,同时系统会在区块链上销毁同样金额的数字资产。

\end{itemize}

\subsection{资产数字化}

基于区块链的智能合约,实现资产的数字化及数字化交易流转。

\begin{itemize}
\item 基于以太坊ERC721等标准协议实现资产数字化,因为资产信息和交易信息在链上,极大的方便了资产的交易流转。

\item 账户即身份,基于账户或多重签名钱包来管理各种资产和执行合约功能。

\item 基于智能合约实现资产快速结算,降本增效,公开透明。

\end{itemize}

\subsection{新的记账方式}

基于区块链智能合约的记账方式代替传统的记账方式。

\begin{itemize}
\item 区块链的底层本身就是分布式账本,数据极难被篡改

\item 基于区块链智能合约的自动化运行,减少人为错误,可以配套风控系统,进一步减少系统风险、提高系统安全

\item 存在多方参与,彼此之间信息不透明,通过区块链的信息实时同步来消除或者减弱信息不对称,从而达成共识

\end{itemize}

\subsection{让一切在阳光下运行}

规则透明、收益透明、分配透明,方便大家监督,特别监管部门的审计与追踪,例如税务部门。

\begin{itemize}
\item 充分利用区块链的去信任化和价值传输的特性,实现多方协作。

\item 运行的智能合约都是开源审计的,其业务逻辑让大家共同审核监督。

\item 所有的商品资产信息、数字资产信息、供应商信息、交易信息、收益、分配、税费等等都是可审计的,而且都基于智能合约的自动化运行,降本增效。

\item 用户的收益、分配都是基于智能合约执行,公开透明。

\item 基于区块链智能合约可以更灵活的支持各种促销规则,比如抽奖促销、满多少减多少等等,杜绝暗箱操作、增加平台扩展性。

比如通过可验证延迟函数(VDF)等技术,可以基于伪随机数生成器生成满足更高的随机性的随机数,应用于抽奖促销等应用场景。

等等……

\end{itemize}


\subsection{开放力:生态参与者共同治理}

开放力是一种制度,尽可能照顾各方生态参与者的发展利益,达成共识。分歧与冲突是难免的,但我们可以通过开放的心态,在生态和谐发展的目标下达成统一共识,多元并包,持续发展。

结合区块链与密码学安全技术,平台有多种工具来让生态参与者表达自己的意见与看法,例如:

\begin{itemize}
\item 基于区块链智能合约的直接投票、投诉与评价

\item 基于盲签名+区块链的匿名投票

结合区块链+FOO电子投票协议(或CGS方案等等)

\item 基于区块链智能合约+可链接的环签名的匿名投票

\item 基于环签名的匿名评价

\item 基于环签名的匿名投诉

等等……
\end{itemize}


\subsection{溯源}

因为交易信息在链上,可以方便产品溯源。一旦出现问题,可以快速溯源寻找问题来源,保障消费者利益。特别是杜绝假冒伪劣产品,只有生产高质量的产品才能振兴中国制造,建立公众消费信心,促进经济发展。

\subsection{满足一定的数据私密性}

\begin{itemize}
\item 以太坊的账户本身具有弱匿名性
\item 混币机制可以实现交易隐私
\item 隐私地址(Stealth Address)可以隐藏去向
\item 环签名(Ring Signatures)可以隐藏来源
\item 机密交易(Confidential Transactions)可以隐藏交易信息
\item Bulletproofs:可以用来实现机密交易,隐藏交易金额
\item ZoKrates:以太坊上的zksnarks(零知识证明算法)工具箱
\item 基于X25519密钥交换协议实现加密通信,保障交易双方的通信安全

等等……
\end{itemize}

%弱匿名性,数据脱敏是指对某些敏感信息通过脱敏规则进行数据的变形,实现敏感隐私数据的可靠保护。基于第三方的身份认证。
%定位服务
\subsection{激励相容}

机制设计理论中“激励相容”是指:在市场经济中,每个理性经济人都会有自利的一面,其个人行为会按自利的规则行为行动;如果能有一种制度安排,使行为人追求个人利益的行为,正好与企业实现集体价值最大化的目标相吻合,这一制度安排,就是“激励相容”。现代经济学理论与实践表明,贯彻“激励相容”原则,能够有效地解决个人利益与集体利益之间的矛盾冲突,使行为人的行为方式、结果符合集体价值最大化的目标,让每个员工在为企业“多做贡献中成就自己的事业,即个人价值与集体价值的两个目标函数实现一致化。

\begin{itemize}
\item 激励

提高人的主观能动性能带来生产力水平的提高。

我们会经历生存驱动、荣誉驱动、物质驱动、到天赋展现并基于个人兴趣爱好驱动,再加上物质是生活的保障,所以我们的初衷就是通过物质、荣誉激励大家,发现并培养热爱这项工作或职业的伙伴,同时提供学习机会,不断提高服务水平。

总之,需要对生态参与者激励,维护生态的健康发展。
\begin{itemize}
\item 消费者消费是要激励的
\item 服务提供者提供服务是要激励的
\item 消费者的评论、投诉、建议等是要激励的(特别是交易和服务相关)
\item 用户参与平台的投票等活动是要激励的

等等……
\end{itemize}

生态参与者有利于平台生态的行为都有激励,激励主要有三种形式:其一就是奖励积分,可以通过积分换Token智能合约得到Token,得到Token就相当于获得分配权,可以转入Token价值合约,参与获取奖励;其二就是荣誉榜;其三就是更好的职业发展机会。

\item 服务提供者需要更大的工作自由度

现在社会普遍加班严重,透支生命力,我们希望通过系统来解决业务的复杂度,方便大多数人都可以胜任这个工作,通过协调每个服务提供者的时间来提供服务,让每个服务提供者有更大的自由度来面对生活中的问题。这也许是共享经济在人力资源的一种实践。

%员工享有更大的自由度,兼职、临时工、可选择,大多数人都可以胜任这个工作,系统来解决复杂性,但提升服务满意度,消费者可以匿名评价或投诉

\item 生态参与者享有一切收入的分配权

\begin{enumerate}

\item 生态参与者共享产品与服务创造的直接价值。

\item 衍生价值共享,比如传统平台拥有消费者数据,利用人工智能、大数据等技术手段创造新的价值模式,但消费者是没有分享到该价值,新的模式下这些利益价值应该生态参与者共享。

\item 其他收入(比如广告、增值服务等)也应该与生态参与者共享。

\end{enumerate}

%评分来源于:评价、销量、热度、浏览量,各种各样的算法
%机制设计里面的,如何让大家积极贡献自己的价值
%根据反馈速度或响应度来评价服务态度。
%用户的消费、评价等等都有激励,可以把币加入激励合约获得激励

\end{itemize}

\section{在各商业领域的应用}

下面只举几个例子,很多场景应用不胜枚举。

\subsection{在餐饮业的应用}

\subsubsection{企业模式}

\begin{enumerate}

\item 发行Token、部署与法币1:1的数字资产合约、部署价值合约、会员相关合约等等。

与法币1:1的数字资产合约方便资产在系统内流转。价值合约是一个价值分配合约,如果生态参与者想参与分配,可以把拥有的Token注入到该合约里,平台的一切可分配的收入也会注入,占用的Token越多,享受的分配越多。会员相关合约是用来管理会员相关服务。

\item 众筹开店,给予Token作为分配权(此项有别于ICO,给予的Token是不能公开流通的)。

部署众筹合约,一但众筹成功,相关投资人可以获取Token,同时系统部署门店合约,指定经营负责人,参与餐厅营运前的所有相关事务,并配置门店基础数据,包括人力资源、供应商、商品与服务等等。

门店负责人需要公开列出所有开支项,比如租金、水电气费、装修费、器材设备费等等,方便大家监督。

\item 会员服务

消费者需要付会员费才能成为会员,才能享受低价与价值激励。会员费会注入价值合约,全生态参与者共享。非会员享受正常零售价格,平台获取的收益全生态参与者共享。

\item 超低价

低盈利水平,扣除食材成本、租金、水电气、人力成本、备用金、税费、运营奖励等等基本开支,可以只有1\%的利润,而且这部分利润注入价值合约,并奖励相应的Token。因为人力成本和水电气成本等不是固定的,我们可以通过月度反结来处理,多余的部分注入价值合约,少的部分可以借用备用金,并提高下月度的基本开支比率。

会员消费,假设成本价(Cost)为$P_{cost}$,售价(Sale Price)为$P_{sale}$,增值税(Value Added Tax)为$6\% \cdot (P_{sale}-P_{cost})$,营运成本(Operating Expenses)为${\alpha}\% \cdot P_{sale}$,门店运营奖励(Store Operation Reward)为$1\% \cdot P_{sale}$,备用流动资金(Liquidity)为${\beta}\% \cdot P_{sale}$,所有会员奖励(Dividend)为$1\% \cdot P_{sale}$,满足:
\[
	P_{cost} + 6\% \cdot (P_{sale}-P_{cost}) + {\alpha}\% \cdot P_{sale} + 1\% \cdot P_{sale} + {\beta}\% \cdot P_{sale} + 1\% \cdot P_{sale} = P_{sale}
\]

则
\[
	P_{sale} = \frac {94}{92 - \alpha - \beta} \cdot P_{cost} 
\]
毛利(Gross Profit) = 营业收入(Total Revenue) - 营业成本(Cost of Revenue),该笔消费毛利(Gross Profit)为$P_{sale} - P_{cost} = \frac{2 + \alpha + \beta}{92 - \alpha - \beta} \cdot P_{cost}$,毛利率(Gross Profit Margin)为$\frac{2 + \alpha + \beta}{94}$。

营运成本主要包括用人成本(即所雇佣人员的工资费用)、工资税和员工福利费、水电费、燃料费、保险费、物料消耗及低值易耗品摊销、折旧费、维修费、财务费、租金、其他费用等等。

\item 门店运营

采购、收货、退货、销售、税费、分配等等所有业务流程都基于智能合约自动运行。

\item 其他

系统提供外卖、半成品零售、配料零售等增值业务。

消费者可以针对产品与服务发表评价、投诉与建议,获得相应的Token激励。

员工也可以获得Token激励,比如基于客户的服务评价、工作时长、销售业绩等等。

所有的规则和制度都是公开的,如果有异议,可以一起协商。

\end{enumerate}

\Emph{现在人们工作繁忙,如果自己回家做饭,买菜洗菜切配菜需要花费很多时间,我们可以统一的预处理,就像现在餐饮业的中央厨房一样,按照统一规格加工成半成品(主要包括食材、调料、配料等等),不仅可以供应给餐厅消费,也可以出售给公众消费,消费者只要回家负责炒菜就可以食用了。}

\Emph{我们可以通过区块链来溯源食材调料+直播技术监督采购、清洗、切配等现场作业,为公众提供放心的食材,也可以通过流动的食材储物柜来方便公众购买。}

\subsubsection{平台模式}

平台模式不同于企业模式的最主要一方面是服务主体不是平台经营,比如外卖。

这时候主要的可分配的盈利来源主要有:

\begin{itemize}
\item 收取服务主体的服务费
\item 交易费,每成交一笔收取相关的费用
\item 广告费
\item 增值业务,比如大数据分析、服务提供方运营统计分析等等
\end{itemize}

消费者消费、评价、投诉、建议等都会获得Token激励,同时服务主体每成交一笔交易也会获得Token激励。

\subsection{其他领域的应用}

这种模式也可以应用于零售业、医疗、电商平台、农产品等领域。

\begin{enumerate}
\item 在医疗领域,比如大型医疗设备,我们可以大家一起众筹,拥有设备的所有权,但经营使用权可以授权给医院,减少患者相关费用开支。同时这种模式也可以用于医药零售,现在药品零售价与出厂价价格相差太大,通过这种模式也降低公众的医疗药品开支。

\item 在农产品领域,通过这种模式,减少中间环节的利益占比,一方面增加农民收入,另一方面减少消费者生活开支。

\item 现在电商平台的广告流量费、交易费等等成本高企,通过这种模式,把收入直接贡献给生产制造商,让企业有资本创新提供更好的产品与服务。

\end{enumerate}

\section{总结}

我们可以加大教育投入,发掘每个人的天赋,培养每个人的兴趣爱好,就会在各领域培养更多更优秀的人才,同时向大众开放创投,大家积极勇于创新创造,然后通过这种模式,让大家勇于消费、敢于消费,提高生活水平与质量,生产企业也才有资本去提升工艺、创造更好的产品与服务,带动经济发展,同时减少贫富差距,更利于社会的和谐发展。整个社会生态繁荣后,就业还是问题吗?经济发展还是问题吗?人们生活水平、生活质量还是问题吗?

\end{document}