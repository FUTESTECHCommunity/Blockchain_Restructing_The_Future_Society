\documentclass[UTF9]{ctexart}
\title{区块链:重构未来社会}
\date{2018年12月5日}
\author{未来工程科技\\邹远春}
\usepackage{xeCJK}
\usepackage{graphicx}
\usepackage{float}
\usepackage{listings}
\usepackage{verbatim}
\usepackage{mathtools}
\bibliographystyle{alpha}

\begin{document}

\maketitle

\newcommand\Emph{\textbf}

\tableofcontents

\section{经济学家的思考与贡献}

\begin{itemize}

\item 2018年诺贝尔经济学奖得主保罗·罗默的“内生经济增长模型”

%\begin{quote}
该模型由三个基本前提或假定:
\begin{itemize}
\item 第一,技术进步是经济增长的核心。
\item 第二,大部分的技术进步源于市场激励而致的有意识的投资行为,即技术是内生的。
\item 第三,创新能使知识成为商品。
\end{itemize}
%\end{quote}

该模型的经济含义和政策结论是:

\begin{itemize}
\item 增长率随着研究的人力资本的增加而增加,与劳动力规模以及生产中间产品的工艺无关。大力投资于教育和研究开发有利于经济增长,而直接支持投资的政策无效。

\item 经济规模不是经济增长的主要因素,而人力资本的规模才是至关重要的。一个国家必须尽力扩大人力资本存量才能实现更快的经济增长。经济落后国家人力资本低,研究投入的人力资本少,增长缓慢,经济将长期处于“低收入的陷阱”。

\item 由于知识的溢出效应和专利的垄断性,政府的干预是必要的。政府可通过向研究者、中间产品的购买者、最终产品的生产者提供补贴的政策以提高经济增长率和社会福利水平。
\end{itemize}

\item 2007年诺贝尔经济学奖得主埃里克·马斯金“机制设计理论”

机制设计的核心是明确目标和动机,机制设计的两个特征:
\begin{itemize}
\item 第一,作为机制设计者,当你面对信息不对称的问题时,可以不通过自己来寻求最优解决方案,而是通过设计一个机制来完成你的目标。
\item 第二,在机制设计过程中,要综合考虑参与者的动机,实现激励相容,达到共赢。
\end{itemize}

20多年前,美国政府想把无线电频率转售给最重视它且能创造最大价值的企业,政府找到这家企业的常规方法有两个:一是政府去询问每家公司的报价,但企业可能会为了拿到许可证而虚报价格,造成不实的高价竞争;另一个方法是竞标,但这样可能会带来低价竞标,因为每个公司都想以有利润差的价格拿到许可证。威廉·维克瑞教授提出的解决方案是:公开竞标,高价者得,但是实际付费只需按照第二高的竞价。马斯金说,在这个机制下,低价竞标会成为一个劣势,因为如果企业把价做的低了,很有可能会流标;同时企业也不会虚高报价,如果虚高报价,很可能最后价格高出实际,做了笔亏本的买卖。

\item 2017年诺贝尔经济学奖得主理查德·塞勒

\begin{itemize}
\item 只要一起合作的人没有占对方便宜,人们总是倾向于选择合作。经济学模型中理性、自利的人基于利己主义做出相互选择,他们往往所得到的只是总体的次优选择,而“明智的合作者”才会得到最优的收益。
\item 《助推》
\item 做出大大小小的决策是我们一生中耗时很多、很费心力的事。
\item 我们往往深陷难以计数的偏见和非理性中,做出荒谬的判断。
\item 事实上,不需要强制性手段,也不需要硬性的规定,助推便能保证你同时收获“很大利益”和“自由选择权”。例如,政府颁布法令禁止人们食用垃圾食品不算助推,把低价的新鲜水果便捷地呈现在人们眼前,让人们主动地选择健康食品,这才是真正的助推。
\item 政府需要用行为经济学方面的知识对政策进行优化设计,引导人们在教育、投资、卫生保健、抵押贷款及环境保护等领域做出让人们更健康、更富有、更快乐的决策,对社会乃至全球很有助益的选择。
\end{itemize}

\item 2016年诺贝尔经济学奖得主安格斯·迪顿

安格斯迪顿最主要的学术贡献在于提供了定量测量家庭福利水平的工具,以此来更准确地定义和测量贫困,对更加有效地制定反贫困政策有着重要意义。

改革开放以来,我国经济持续快速增长,城乡居民收入水平不断提高,但收入差距也在不断扩大,收入分配不合理、不公平的现象愈加突出,这将严重影响经济的持续增长、影响社会的和谐稳定。在这种背景下,准确测量家庭福利水平以及社会整体的不平等和贫困程度,进而制定更有针对性的公共政策来降低不平等、减少贫困就显得至关重要。

\end{itemize}

\section{我对当前经济发展的思考}

人类社会不断的创新发展让生产力得到巨大的提升,提高了人们的生活水平,从茹毛饮血到如今的物质文化繁荣;生产关系经历原始社会、奴隶社会、封建社会、资本主义社会也得到不断的改进。但是我们现在社会还是有本身存在的一些问题,比如经济危机,我们耳闻所见的一则小故事:

\begin{quote}
\Emph{寒冷的北风呼啸着,一个穿着单衣的小女孩蜷缩在屋子的角落里。}

\Emph{“妈妈,天气这么冷,你为什么不生起火炉呢?”小女孩在瑟瑟发抖。}

\Emph{妈妈叹了口气,说:“因为我们家里没有煤。你爸爸失业了,我们没有钱买煤。”}

\Emph{“妈妈,爸爸为什么失业呢?”}

\Emph{“因为煤太多了。”}
\end{quote}

而我们现实生活中,一边大量的蔬菜水果卖不出去,菜农果农哭泣;一边超市高价的蔬菜水果让大众消费。

我在想这到底是生产力不足、生产过剩还是消费不起?所以我觉得在生产关系领域还有很多的改善空间,不要生产力上去了,生产关系还落后着,这时贫富差距加大是必然的。

让一部分人先富起来,带动大部分地区,然后达到共同富裕,结果让一部分人先富起来实现了,其他的都遗忘了。


\begin{quote}
\Emph{十一届六中全会提出我国所要解决的主要矛盾是人民日益增长的物质文化需要同落后的社会生产之间的矛盾。}
\end{quote}

通过改革开放,中国劳动力生产力得到巨大提升;恢复高考改革教育,也带来知识生产力的提升;看重科研与创新,也带来科技生产力的提升。但中国经济目前取得的成就主要是依靠劳动力生产力,释放所谓的人口红利,主要依靠广大的农名工与自然生态环境资源。

社会是不断变化的。

\begin{quote}
\Emph{十九大报告提出我国社会主要矛盾已经转化为人民日益增长的美好生活需要和不平衡不充分的发展之间的矛盾。}
\end{quote}

我们经常谈经济发展,为了什么?赚钱?提升GDP?升官发财?我认为发展是为了保障民生,提高人民生活水平。

\begin{enumerate}
\item 我们重生产制造还是重营销流通?
\item 什么是基本民生保障?或应该包含些什么?
\item 生产力的提升靠劳动力生产力、知识生产力、科技生产力?
\item 重资本财富还是实体经济?
\end{enumerate}

当前中国社会在电商、共享经济领域的模式创新已经走在世界前列,极大的方便了我们的生活。但是我们在实体经济领域呢?实体经济需要更多的资本、更多的创新才能发展,从而保障民生,提高人民生活水平。

\begin{quote}
\Emph{实体经济,指一个国家生产的商品价值总量。是人通过思想使用工具在地球上创造的经济。包括物质的、精神的产品和服务的生产、流通等经济活动。包括农业、工业、交通通信业、商业服务业、建筑业、文化产业等物质生产和服务部门。也包括教育、文化、知识、信息、艺术、体育等精神产品的生产和服务部门。实体经济始终是人类社会赖以生存和发展的基础。}
\end{quote}

\Emph{生产力}

劳动力生产力、知识生产力与科技生产力,我们应该着重在知识生产力和科技生产力,所以要搞好教育和创新,保障实体经济的创新发展。

但中国目前还是看重劳动力生产力比较多一些吧,加班加点这么多,靠加班加点的劳动力生产力能带来巨大的生产力提升吗?如果身体变差了,他们如何生活?如何养育一家老小?其他的医疗保障?养老保障?是不是讽刺的觉得人病了,老了又可以大力发展医疗产业、养老产业,GDP又有发展保障了?

创新不是说创新就能立马出成果的,需要去思考去学习,急功近利靠人头不是又好又快,更可能的是事倍功半。

\Emph{生产关系}

生产力提升了,但是还是保障不了民生,消费不起?所以生产关系也要同步跟上。因为生产关系决定了分配,只有合理的分配才能保障民众有能力去消费,才能更进一步促进实体经济的创新发展,才能更普遍的保障民生,提高人民生活水平。

除了消费能给生产制造实体经济领域带来收益,金融也要为实体经济服务,因为实体经济创新发展需要大量的资本投入。

把创新作为发展的核心引擎,保障教育,通过改革生产关系来改善分配,让大家敢于消费,从而保障民生,提高人民生活水平。


就比如老龄化养老问题,仅仅靠多生人口吗?那日常教育生活支出压力呢?更多的是应该关注民生,关注民众健康,并通过人工智能、物联网等技术来辅助养老护理,只有健康的身体才是最好的养老资本,才能好好的过完后半辈子,才能去世界到处去看看,减少对整个社会的压力。

\section{区块链带来的思考}

我们一谈到区块链,很多会想到比特币,会想到炒币。但是区块链更代表的是一系列技术的统称,就像互联网或移动互联网。

很多人说区块链带来的是生产关系的革命,可以进一步解放生产力,我对这个观点是比较赞成的。

区块链不仅仅是生产关系领域工具,它同时也可以作为信用价值桥梁,实现信用传递,从而保障良好的社会诚信体系,良好的诚信可以降低社会运行成本。

我会在后面的一系列文章中讲区块链的技术与应用,主要在三个部分。

\subsection{第一部分 区块链在体制改革中的作用(略)}

\subsection{第二部分 区块链在经济领域的应用}

通过区块链来改善目前生产制造与民众消费的脱节,减少营销流通环节对生产制造的利益侵占(特别是物流、平台、中介、中间商等领域),让生产制造得到更多的利益,不仅保障了实体经济的创新发展,从而又降低了民众的消费支出,带来更多的消费。

这里只是大纲,后面会详细的论述如何基于区块链技术来构建。

\begin{itemize}

\item 一种泛商业形态的商业创新模式及其在餐饮、零售、农产品、医药、电商平台、平台经济(共享经济、租恁经济)等领域的应用

\item 对教育的支持

教育领域可以基于区块链成立职业教育公益基金等各种教育公益基金,监管各项开支。

辅助教育改革,实现孩子的全面发展,在知识教育、能力(创新、批判、沟通、合作等)、精神(探索精神、契约精神、企业家精神、意志力、专注力、担当、人格、人文情怀等)、身体各方面齐头并进。

\item 对创新与资本市场支持

投资有风险,但实体经济,特别是对创新水平有要求的企业需要大量资本。大众消费优秀的产品是对生产制造企业最好的支持,但在企业没有生产创造产品之前需要持续的投资,国家应该有平台来跟踪这些项目,然后吸引民间资本投资。虽然每个人的资本贡献是小的,但组合起来的资本贡献是巨大的;虽然项目失败的风险很高,但每个个体风险是可控的。一旦项目成功,每个个体都可以享受成果。

\item 对农村经济的支持

对农村土地承包流转、农产品物流、医保社保跟踪管理、农村集体经济(股权、所有权、分红等)、各种款项跟踪监督(特别扶贫款项等)、政务公开、涉及农村的工程(比如路建等)、当地招商投资政策、各种资源环境跟踪(有什么特产?适合种植什么?方便根据当地特色因地制宜制定经济发展策略)、农业机械化发展、农业科技人才、农业保险(防止天灾等自然灾害因素导致的收入不稳定,菜贱伤农,菜贵伤民)、一些创新商业模式(比如城市居民可以投资农产品生产,不仅农民收入有一定的保障,而且可以享受新鲜的农产品或农家乐等服务)、一些其他科技支持(比如最优传输、农业物联网等等)等等。释放农村经济活力,实现当地人可以当地(或就近)就业。

\item 其他领域的支持

区块链可以应用更多领域,金融科技(清算、支付、金融保险等等)、公益、彩票、房地产、医保社保(方便人才流通)等等。

\end{itemize}

\subsection{第三部分 区块链技术的发展展望}

比特币、以太坊、Algorand、TrueBit、Plasma、POW、POS、DPOS、DAG、VRF、VDF、RSA Accumulators、ZK-SNARKs、ZK-STARKs、Bulletproofs、形式化证明、跨链等项目或技术的发展让我们看到基础技术在不断完善,性能提高、安全性增强、成本降低,然后才能支撑更大的应用市场。就像互联网一样,操作系统、网络技术等基础技术的不断完善才带来互联网生态的爆发式增长。

\end{document}